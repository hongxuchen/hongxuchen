%# -*- coding:utf-8 -*-

\documentclass[11pt,a4paper,colorlinks=true]{moderncv}

\usepackage{fontspec,xunicode}
\setmainfont{Tahoma}
\usepackage[slantfont,boldfont]{xeCJK}
\usepackage{xcolor}
\usepackage{pifont}

%\usepackage[
%%backend=biber, 
%natbib=true,
%style=numeric,
%sorting=none
%]{biblatex}
%\addbibresource{pubs.bib}
%\usepackage{biblatex}


\setmainfont{Times New Roman}%缺省英文字体.serif是有衬线字体sans serif无衬线字体
\setCJKmainfont[ItalicFont={KaiTi}, BoldFont={SimHei}]{STZhongsong}%衬线字体 缺省中文字体为
\setCJKsansfont{STZhongsong}
\setCJKmonofont{STZhongsong}%中文等宽字体
% \setCJKsansfont{STSong}
% \setCJKmonofont{STFangsong}%中文等宽字体
%-----------------------xeCJK下设置中文字体------------------------------%
\setCJKfamilyfont{song}{SimSun}                             %宋体 song
\newcommand{\song}{\CJKfamily{song}}
\setCJKfamilyfont{fs}{FangSong_GB2312}                      %仿宋2312 fs
\newcommand{\fs}{\CJKfamily{fs}}
\setCJKfamilyfont{yh}{Microsoft YaHei}                    %微软雅黑 yh
\newcommand{\yh}{\CJKfamily{yh}}
\setCJKfamilyfont{hei}{SimHei}                              %黑体  hei
\newcommand{\hei}{\CJKfamily{hei}}
\setCJKfamilyfont{hwxh}{STXihei}                                %华文细黑  hwxh
\newcommand{\hwxh}{\CJKfamily{hwxh}}
\setCJKfamilyfont{asong}{Adobe Song Std}                        %Adobe 宋体  asong
\newcommand{\asong}{\CJKfamily{asong}}
\setCJKfamilyfont{ahei}{Adobe Heiti Std}                            %Adobe 黑体  ahei
\newcommand{\ahei}{\CJKfamily{ahei}}
\setCJKfamilyfont{akai}{Adobe Kaiti Std}                            %Adobe 楷体  akai
\newcommand{\akai}{\CJKfamily{akai}}


%------------------------------设置字体大小------------------------%
\newcommand{\chuhao}{\fontsize{42pt}{\baselineskip}\selectfont}     %初号
\newcommand{\xiaochuhao}{\fontsize{36pt}{\baselineskip}\selectfont} %小初号
\newcommand{\yihao}{\fontsize{28pt}{\baselineskip}\selectfont}      %一号
\newcommand{\erhao}{\fontsize{21pt}{\baselineskip}\selectfont}      %二号
\newcommand{\xiaoerhao}{\fontsize{18pt}{\baselineskip}\selectfont}  %小二号
\newcommand{\sanhao}{\fontsize{15.75pt}{\baselineskip}\selectfont}  %三号
\newcommand{\sihao}{\fontsize{14pt}{\baselineskip}\selectfont}         %四号
\newcommand{\xiaosihao}{\fontsize{12pt}{\baselineskip}\selectfont}  %小四号
\newcommand{\wuhao}{\fontsize{10.5pt}{\baselineskip}\selectfont}    %五号
\newcommand{\subwuhao}{\fontsize{10pt}{\baselineskip}\selectfont}    %次五号
\newcommand{\xiaowuhao}{\fontsize{9pt}{\baselineskip}\selectfont}   %小五号
\newcommand{\liuhao}{\fontsize{7.875pt}{\baselineskip}\selectfont}  %六号
\newcommand{\qihao}{\fontsize{5.25pt}{\baselineskip}\selectfont}    %七号

\newcommand{\ccfe}{\underline}
\newcommand{\ptype}[1]{}


%\usepackage{fontawesome}
% \setCJKmainfont[BoldFont={WenQuanYi Micro Hei/Bold}]{WenQuanYi Micro Hei}
%\defaultfontfeatures{Mapping=tex-text}
%\XeTeXlinebreaklocale "zh"
%\XeTeXlinebreakskip = 0pt plus 1pt minus 0.1pt
% moderncv themes
\moderncvtheme[blue]{classic}                 % optional argument are 'blue' (default), 'orange', 'red', 'green', 'grey' and 'roman' (for roman fonts, instead of sans serif fonts)
%\moderncvtheme[green]{classic}                % idem
%\moderncvtheme[blue,roman]{hht}
% character encoding



% adjust the page margins
\usepackage[scale=0.93]{geometry}
%\setlength{\hintscolumnwidth}{3cm}                                             % if you want to change the width of the column with the dates
%\AtBeginDocument{\setlength{\maketitlenamewidth}{6cm}}  % only for the classic theme, if you want to change the width of your name placeholder (to leave more space for your address details
\AtBeginDocument{\recomputelengths}                     % required when changes are made to page layout lengths

% personal data
\firstname{陈泓旭}
\familyname{}
\title{网络安全, 软件工程}               % optional, remove the line if not wanted
% \address{1989/08/06}{}    % optional, remove the line if not wanted
% \mobile{13027944806}                    % optional, remove the line if not wanted
\email{hongxu\_chen@foxmail.com}                     % optional, remove the line if not wanted
% \homepage{https://github.com/hongxuchen} % optional, remove the line if not wanted
\social[github]{HongxuChen}
% \social[twitter]{hongxuchen}
\social[linkedin]{hongxu-chen-ntu}
% \extrainfo{%
%   wechat: hongxu\_chen
% }

% \photo[64pt][0pt]{400x514.jpg}                         % '64pt' is the height the picture must be resized to and 'picture' is the name of the picture file; optional, remove the line if not wanted
\photo[64pt][0pt]{me.png}                         % '64pt' is the height the picture must be resized to and 'picture' is the name of the picture file; optional, remove the line if not wanted
%\quote{陈泓旭}                 % optional, remove the line if not wante

%\nopagenumbers{}                             % uncomment to suppress automatic page numbering for CVs longer than one page

\setlength{\hintscolumnwidth}{3.2cm}

\renewcommand\refname{论文及获奖}

%----------------------------------------------------------------------------------
%            content
%----------------------------------------------------------------------------------
\begin{document}
\maketitle
% \vspace*{-10mm}

\section{工作经历}
\cventry{2020.10至今}{华为}{主任工程师}{}{}{以\href{https://career.huawei.com/reccampportal/portal5/topminds.html}{天才少年}身份加入华为新加坡研究所,于2021年7月调回2012实验室-可信实验室(深圳)。
\begin{enumerate}
	\item 研究华为软件实现和软件设计中一致性看护能力,通过软件设计中的安全性、可维护性规约识别设计架构中接口和实现的质量腐化问题。在此期间我带领团队实现了\textit{代码和架构一致性看护平台TENET},通过自研C/C++、Java架构逆向引擎,识别代码实现中的关键架构元素,并结合规约完成一致性检查;相关能力已落地华为终端BG、海思、车BU等产业。
\item 基于前述TENET工作,进行\textit{代码分析的数字化服务}项目研究,完成C/C++、Java、ArkTS的代码知识图谱的设计和实现。在技术上攻关单仓3000w+的C/C++源代码级精准分析能力。该项目已孵化形成{\ding{172}}代码质量度量和架构异味识别引擎用于Clean Code L3检查的IDE端和版本级检查平台CleanArch,落地于公司13个产业的80+个业务;{\ding{173}}基于代码知识图谱和LLM的精准代码文档生成和智能问答的Web端类deepwiki服务,已完成对公司内410+存量代码仓的分析。
\item 研究软件缺陷和漏洞、缺陷代码测试集、检查工具的映射关系,建立类CWE的代码缺陷字典,通过对软件设计、编码、代码review、静态检查、DT测试等软件开发过程进行系统性分析,形成基于数据驱动的代码缺陷主动防御体系。该工作正联合华为公司数通、光、终端BG等主要产业进行能力共建,目前着重解决内存安全类缺陷问题。
\end{enumerate}
}
\cventry{2019.08-2020.9}{南洋理工大学}{Research Fellow}{}{}{扩展了自研模糊测试框架FOT, 项目主页\url{https://sites.google.com/view/fot-the-fuzzer},{\ding{172}}增强了其对结构化输入的支持,提升了测试用例生成的合法性;{\ding{173}}对多线程分析场景下提供了模糊测试方案MUZZ,提升了非顺序执行场景下的模糊测试有效性。同时进行的项目包括高性能跨CPU二进制模糊测试框架BiFF,完成了在ARM、x86及RISCV下仅二进制场景下的高效fuzz,相关工作已落地Continental渗透测试。 }
\cventry{2014.05-2015.08}{南洋理工大学}{Research Associate}{}{}{基于LLVM的数据流分析, 该项目主要为了精化数据流分析来指导动态测试的有效性。}
\cventry{2013.02-2013.11}{微软亚洲研究院}{研究实习生}{}{}{通过静态分析提高白盒测试程序补丁的有效性; 实现了对补丁程序切片静态代码分析工具, 使得被切片后的程序可以有效地用于白盒测试,相关工作成为SRG组内研究成果。}
\section{教育经历}
\cventry{2015.08-2019.07}{博士}{南洋理工大学}{计算机科学与技术}{网络安全实验室, 导师: \href{https://personal.ntu.edu.sg/yangliu}{LIU Yang教授}}{主要关注移动系统安全和软件的实现安全:}
\cvlistitem{系统安全: 利用类型系统验证依赖于权限的信息流安全, 该类型系统主要对Android系统的App之间的交互进行建模, 我设计了该类型系统并完成对系统健壮性的形式化验证。}
\cvlistitem{软件安全: 从事对C/C++程序进行模糊测试的研究, 主要思路为利用程序分析的方法来提高模糊测试的有效性。}
\cventry{2011.09-2014.03}{硕士}{上海交通大学}{计算机科学与技术}{高可靠软件实验室, 导师: \href{https://stap.ait.kyushu-u.ac.jp/~zhao/}{赵建军教授}}{关注基于程序分析的软件可靠性研究, 包括基于LLVM的Andersen指针分析和程序切片,基于KLEE的符号执行优化,基于SOOT框架的单元测试有效性分析等。}
\cventry{2007.09-2011.07}{本科}{南京理工大学}{信息与计算科学}{}{}
\section{科研项目}
% \subsection{科研项目}
\cvline{FOT}{2017.07-2020.09\quad 主导开发并维护\textit{模糊测试框架FOT}, 该框架由Rust编写,强调使用静态分析来指导模糊测试。FOT已经检测了100+开源软件并从其中找到了300+漏洞, 61个被赋予CVE编号, 这包括GNU libc, FFmpeg, ImageMagick等知名开源项目中的10个严重或高危漏洞。 具体漏洞详见\url{https://github.com/fot-the-fuzzer/pocs}. FOT获NASAC2017原型竞赛(命题型)一等奖, 并被CCF-A类会议ESEC/FSE 2018接受。基于该框架的Hawkeye和Cerebro分别被CCF-A类会议CCS 2018和ESEC/FSE 2019接受。}
\cvline{BiFF}{2018.11-2020.09\quad 参与开发\textit{高性能跨CPU模糊测试框架BiFF}, 该框架致力于改进现有二进制模糊测试技术, 使用轻量级的hooking技术及对``服务型''程序的测试流程优化,提高对不同CPU下IoT设备模糊测试的有效性。该框架获得NASAC2019原型竞赛(自由型)一等奖。}
\cvline{MUZZ}{2018.07-2019.07\quad 设计并实现了\textit{多线程场景下的模糊测试技术MUZZ},结合传统静态分析技术实现了面向多线程下覆盖率的插桩、记录多线程上下文的插桩和用于干预多线程调度器的插桩,并在动态模糊测试过程中提出了针对多线程环境的种子选择策略和自适应重复执行策略,提升了模糊测试的有效性和优效性。}
\cvline{Hawkeye}{2017.12-2018.05\quad 设计并提出导向性模糊测试技术Hawkeye, 指出了导向性模糊测试的4个属性及解决方案,并用实验论证了有效性。 该技术被CCF-A类会议CCS 2018接受。}
\cvline{STAndroid}{2015.08-2017.06\quad 该项目立足于Android系统的权限系统, 利用\textit{类型系统}验证依赖于权限的信息流安全; 我设计并完成了对类型系统健壮性的形式化验证, 并实现了基于此的信息泄露检测工具原型。该工作被CCF-B类会议CSF 2018接受。}
\cvline{RBScope}{2013.02-2013.11\quad 该项目关注利用静态分析的方法加强对新程序补丁处的\textit{符号执行}; 主要思想是利用程序切片的方法去除和补丁不相关的程序片段, 从而使测试关注于和补丁相关部分并减小测试状态空间爆炸问题。 我实现了基于LLVM的Andersen指针分析算法和对补丁的\textit{程序切片}。}
%\subsection{开源项目}
%\cvline{markvis}{在 markdown 中直接生成可视化图表的插件
%  \emph{https://markvis.js.org} \textbf{GitHub 1000 stars}}
%\cvline{netjsongraph.js}{用力导向图可视化出无线路由图谱数据 \emph{https://github.com/netjson/netjsongraph.js}}
%\cvline{typing}{Hexo 静态博客主题 \emph{https://github.com/geekplux/hexo-theme-typing}}
%\cvline{UnityVis}{Unity 中的基本可视化图表 \emph{https://github.com/geekplux/Basic-Visualization-in-Unity}}

\section{技能}
\cvline{\textbf{精通}}{静态分析, 模糊测试, 符号执行, LLVM/Clang, C/C++, Java, Python, Rust, Bash, Lua}
\cvline{\textbf{熟练}}{程序语言理论, 编译原理, 形式化验证, 二进制安全, Linux编程, JVM, SMT约束求解}
\cvline{\textbf{了解}}{Go, Kotlin, OCaml, Haskell, Coq, Isabelle}

%\section{论文评审经历}

\section{助教经历}
\cvline{面向对象设计编程}{2018年秋学期\quad 负责面向对象设计与编程(Object Oriented Design and Programming, CE/CZ2002)实验设计、答疑及批改。}
\cvline{软件工程}{2018年春学期\quad 负责软件工程(Software Engineering, CE/CZ2006)实验指导、答疑及批改。}
\cvline{软件系统分析设计}{2017年秋学期\quad 负责软件系统分析与设计(Software Systems Analysis and Design, CE/CZ3003)实验设计及答疑。}
\cvline{计算机安全}{2017年秋学期\quad 负责计算机安全(Computer Security, CE/CZ4062)课程设计及批改。}
\cvline{密码学与网络安全}{2017年春学期\quad 负责密码学与网络安全(Cryptography and Network Security, CE/CZ4024)课程作业设计及批改。}
\cvline{数据结构与算法}{2016年秋学期\quad 负责数据结构与算法(Algorithms, CE/CZ2001)实验设计、答疑及批改。}
\cvline{编译技术}{2016年春学期\quad 负责编译技术(Compiler Techniques, CE/CZ3007)实验设计、答疑及批改。}

% \renewcommand{\listitemsymbol}{-} % change the symbol for lists

% \section{Extra 1}
% \cvlistitem{Item 1}
% \cvlistitem{Item 2}
%\cvlistitem[+]{Item 3}            % optional other symbol% XeLaTeX can use any Mac OS X font. See the setromanfont command below.
% Input to XeLaTeX is full Unicode, so Unicode characters can be typed directly into the source.

% The next lines tell TeXShop to typeset with xelatex, and to open and save the source with Unicode encoding.

%\section{Extra 2}
%\cvlistdoubleitem[\Neutral]{Item 1}{Item 4}
%\cvlistdoubleitem[\Neutral]{Item 2}{Item 5}
%\cvlistdoubleitem[\Neutral]{Item 3}{}

%% Publications from a BibTeX file
\nocite{*}
\bibliographystyle{unsrt}
\bibliography{pubs}
%\printbibliography[title={Peer-Reviewed Journal articles}]


\end{document}
